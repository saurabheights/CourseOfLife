%% The MIT License (MIT)
%%
%% Copyright (c) 2015 Daniil Belyakov
%%
%% Permission is hereby granted, free of charge, to any person obtaining a copy
%% of this software and associated documentation files (the "Software"), to deal
%% in the Software without restriction, including without limitation the rights
%% to use, copy, modify, merge, publish, distribute, sublicense, and/or sell
%% copies of the Software, and to permit persons to whom the Software is
%% furnished to do so, subject to the following conditions:
%%
%% The above copyright notice and this permission notice shall be included in all
%% copies or substantial portions of the Software.
%%
%% THE SOFTWARE IS PROVIDED "AS IS", WITHOUT WARRANTY OF ANY KIND, EXPRESS OR
%% IMPLIED, INCLUDING BUT NOT LIMITED TO THE WARRANTIES OF MERCHANTABILITY,
%% FITNESS FOR A PARTICULAR PURPOSE AND NONINFRINGEMENT. IN NO EVENT SHALL THE
%% AUTHORS OR COPYRIGHT HOLDERS BE LIABLE FOR ANY CLAIM, DAMAGES OR OTHER
%% LIABILITY, WHETHER IN AN ACTION OF CONTRACT, TORT OR OTHERWISE, ARISING FROM,
%% OUT OF OR IN CONNECTION WITH THE SOFTWARE OR THE USE OR OTHER DEALINGS IN THE
%% SOFTWARE.

% !TeX program = lualatex
% The font could be set to Windows-specific Calibri by using the 'calibri' option
\documentclass[]{mcdowellcv}
\usepackage{lmodern}
% For mathematical symbols
\usepackage{amsmath}
\usepackage{url}
\usepackage[svgnames]{xcolor}
\usepackage[colorlinks=true, linkcolor=Maroon, urlcolor=Maroon, citecolor=blue]{hyperref}
\usepackage[symbol]{footmisc}


\renewcommand{\thefootnote}{\fnsymbol{footnote}}


\usepackage[backend=biber,sorting=none]{biblatex}
\addbibresource{main.bib}


% Set applicant's personal data for header
\name{Saurabh Khanduja}
\email{khandujasaurabh@gmail.com}
\github{https://github.com/saurabheights}
\linkedin{https://linkedin.com/in/saurabheights}
\address{Phone: +49 1772075019 \linebreak Address: Willi-graf-strasse 9, Room 337\linebreak Munich 80805}
\contacts{}


\begin{document}

    % Print the header
    \makeheader
    
    \begin{cvsection}{Summary}
    \textit{Computer Vision specialist with 5+ years professional experience, including 2 years devoted to developing real-time vision modules for national defense and 2 years to building image and video processing service for a social media platform. I specialized in neural network interpretatibility and its applications to weakly supervised semantic segmentation as part of my master's degree. As a work student at Terraloupe, I designed a training data management tool for geospatial annotations and at Edge Case Research I designed new SPIs for measuring safety of an autonomous vehicle perception system.}
    \end{cvsection}
    

    % work experience
    \begin{cvsection}{Professional Experience}

            \begin{cvsubsection}{ML Engineer}{Edge Case Research Gmbh}{May 2020 -- Present}
        \textit{Improving Core ML Products and Production Infrastructure (Python, Go, Kubernetes)}
            \begin{itemize}
                \item Designed 4 new safety performance indicators to measure safety of autonomous vehicles.
                \item Designed benchmark to evaluate internal defect predictors on real data as well as synthetically introduced defects.
                \item Integrated serving and testing of pytorch models and integrated tracking metrics to benchmark multi-object tracking associators.
            \end{itemize}
        \end{cvsubsection}
    
        \begin{cvsubsection}{ML Engineer}{Terraloupe Gmbh}{ April 2018 - April 2020}
            \textit{Training Data Management Tool (Python, Django, Postgres)}
            \begin{itemize}
                \item Development of postgres-based inventory database.
                \item Development of inventory query service with support for filtering over data distribution to extract desired subset for training neural networks.
            \end{itemize}
                
            \textit{Core Deep Learning Pipeline (Python, Keras, Tensorflow)}
            \begin{itemize}
                \item Development of single pipeline supporting Image Classification, Object Detection and Semantic segmentation tasks.
                \item Optimization of training and inference phase with resulting GPU and CPU utilization to over $90\%$.
                \item Object based metrics implemented for semantic segmentation.
            \end{itemize}
        \end{cvsubsection}
        
        \begin{cvsubsection}{Image Team Lead}{Roposo, India}{Nov 2015 -- Oct 2017}        
        \textit{Image Processing Service Development (Java, AWS, OpenCV, ffmpeg)}
            \begin{itemize}
                \item Optimizations to efficiently use servers, leading to 70\% cost reduction.
                \item Enable dynamic compression using SSIM metric increasing user retention by 40\%.
                \item Quantifying social posts ownership by measuring plagiarism using image metadata analysis.
                \item Implemented Beauty filter based on variational approach by Farbman, Zeev, et al. \cite{farbman2008edge}
            \end{itemize}
        \end{cvsubsection}
          
        \begin{cvsubsection}{Software Development Engineer}{Amazon, India}{Oct 2014 -- Nov 2015}        
        \textit{Resell Product Form (Java, DynamoDB)}
            \begin{itemize}
                \item Development of data model to reduce user interaction for form completion and data payload for mobile usage.
                \item Improved form completion rate by 500\% with reduction of data payload by 99.5\%.
            \end{itemize}
        \end{cvsubsection}

        \begin{cvsubsection}{Software Development Engineer}{KritiKal Solutions Pvt. Ltd., India}{June 2012 -- Oct 2014}
            \textit{Atmosphere Turbulence Removal Module (C++, Cuda, DLib, OpenCV, Qt)}
            \begin{itemize}
                \item Development of Atmosphere Turbulence Removal Module based on Non-Rigid Registration method \cite{rueckert1999nonrigid}.
                \item Optimizations added using libpthread and cuda, achieving 240x speedup.
            \end{itemize}
            
            \textit{Wide Area Tracking Module (C++, OpenCV, Qt)}
            \begin{itemize}
                \item Designed and implemented a module for controlling Pan and Tilt Device.
                \item Real time stitching of multiple CCD/Infrared Cameras to produce a wider view.
                \item Change Detection and Tracking module to detect and track objects of interest.
            \end{itemize}
        \end{cvsubsection}
    \end{cvsection}
    
    \newpage  % New page
    
        % academic projects
    \begin{cvsection}{Academic Projects}
    \begin{cvsubsection}[2]{Master Thesis}{CAMP\footnote{\href{http://campar.in.tum.de/Chair/ResearchIssueComputerVision}{http://campar.in.tum.de/Chair/ResearchIssueComputerVision}}, Technical University of Munich}{Dec 2020 -- June 2021}
    \textit{Weakly Supervised Semantic Segmentation using Low-level neural network features}
        \begin{itemize}
            \item We proposed a novel approach of dissecting classification neural networks to extract semantic maps.
            \item The method improves mIOU metric by over 20\% on vgg and resnet backbone models w.r.t. class-activation maps.
        \end{itemize}
    \end{cvsubsection}
    
    \begin{cvsubsection}{Research Assistance Project}{Technical University of Munich}{Dec 2019 -- May 2020}
    \textit{Neural Response Interpretation through the Lens of Critical Paths, CVPR 2021 (Pytorch, python) \cite{khakzar-2021}}
        \begin{itemize}
        \item Improving interpretability using path selection via
neurons’ contributions to the response.
        \item Accepted at CVPR 2021
        \end{itemize}
    \textit{Killing Fusion, CVPR 2017 (C++, Matlab) \cite{slavcheva2017killingfusion}}
        \begin{itemize}
        \item Use Killing energy, Level-set energy and data energy to provide non-rigid registration between RGBD frames.
        \end{itemize}
    \textit{Scan completion and semantic segmentation (Python, Pytorch, hdf5)}
        \begin{itemize}
        \item Multi-task learning to improve scan completion of RGBD Voxel Grid of indoor scenes.
        \item We proposed additional learning of semantic information loss will improve scan completion of indoor scenes.
        \end{itemize}
    \end{cvsubsection}
    
    \begin{cvsubsection}{Student Tutor}{Technical University of Munich}{Oct 2018 -- March 2019}    
        \textit{Machine Learning (IN2210) \footnote{\href{https://www.in.tum.de/en/daml/home/}{https://www.in.tum.de/en/daml/home/}}}
            \begin{itemize}
                \item The course is offered to master students at TUM and is attended by more than 500 students. 
                \item Involved in creating assignments for the course and helped students with the homework. 
            \end{itemize}
        \end{cvsubsection}
    
    \end{cvsection}

    \begin{cvsection}{Education}
        \begin{cvsubsection}{Munich, Germany}{Technical University of Munich}{Oct 2017 -- Present}
            \begin{itemize}
                \item M.Sc. in Informatics, GPA: ~1.4/1.0
                \item Graduate Coursework: Machine Learning; Multiple View Geometry; Deep Learning; Protein Prediction;Tracking and Detection in Computer Vision; 3D Scanning \& Motion Capture; Principles of Computer Vision.
            \end{itemize}
        \end{cvsubsection}
        \begin{cvsubsection}{Dhanbad, India}{Indian Institute of Technology}{July 2008 -- May 2012}
            \begin{itemize}
                \item B.Tech. in Computer Science and Engineering,  CGPA: 7.51/10.0
                \item Undergraduate Coursework: Object Oriented Programming; Data Structure and Algorithms; Theory of Computation; Operating System; Computer Networks; Computer Architecture and Digital Image Processing.
            \end{itemize}
        \end{cvsubsection}
    \end{cvsection}
    
    \begin{cvsection}{Technical Skills}
        \begin{cvsubsection}{}{}{}    
            \begin{itemize}
                \item Programming Languages: C, C++, Python, Java, Matlab.
                \item ML/CV Toolkits: PyTorch, Keras, Tensorflow, OpenCV, Numpy, SciPy, Rasterio, ffmpeg, Dlib.
                \item Application Developement: REST, Django, Neo4j, PostgreSQL, MongoDB, AWS, Docker and Kubernetes.
            \end{itemize}
        \end{cvsubsection}
    \end{cvsection}
    
    \begin{cvsection}{Additional Experience and Awards}
        \begin{cvsubsection}{}{}{}    
            \begin{itemize}
                \item Achieved All India rank 1159 in All India Engineering Entrance Exam, 2008 out of 950,000 participants.
                \item Hiring Experience and leading Image Team at Roposo.
                \item Open source contributions: torchvision, metadata-extractor and imantics\footnote{\href{https://resume.github.io/?saurabheights}{https://resume.github.io/?saurabheights}}.
            \end{itemize}
        \end{cvsubsection}
    \end{cvsection}
    
        \begin{cvsection}{References}

         \printbibliography[heading=none]

    \end{cvsection}
    
\end{document}

